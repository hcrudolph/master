As shown in section~\ref{ssec:verif-summary}, the work done to model the \gls{prins} protocol for \textsc{ProVerif} as part of this thesis allows for the successful verification of several intended security properties.
Nevertheless, there is room for improving and expanding on the formal verification with \textsc{ProVerif} in order to validate protocol aspects that have not been covered conclusively.

Firstly, tuning of the resolution strategy employed by \textsc{ProVerif} to derive facts from available clauses may speed-up scenarios that did not terminate using the available default strategies.
The program's user manual describes how facts can be assigned a custom weight (\cite{blanchet2020proverif}, pp.~112-114), guiding the resolution on what facts to prefer or ignore.
While it is not guaranteed that this optimization alone would resolve the issue of non-termination for the concerned queries, it is a refinement step in the verification that has not been tried so far.

Secondly, further analyzing the attack traces produced by \textsc{ProVerif} for the queries that have been explicitly disproven would likely yield additional insights based on the work already performed.
Whether it is optimizing the generation of visual attack graphs using a different toolset or manually retracing the text-based description generated by \textsc{ProVerif}, these steps not performed due to the time limits of this thesis could help getting the most out of the results obtained so far and potentially guide further model improvements or even protocol amendments.

Lastly, certain existing security queries could be refined to produce more meaningful results.
The latest \textsc{ProVerif} version 2.02 released in July 2020 adds support for specifying time points in correspondence assertions which could be used to more accurately verify some of the queries in section~\ref{sec:verification}.
For example, the ones involving rogue \gls{ipx} providers and subsequent validation of their message modifications could benefit from the ability to express temporal relations to either confirm or clear up the discovery of a practical attack.
The only reason why verification has not been attempted with this newer version of the software is that \textsc{ProVerif} 2.02 introduces breaking changes that lead to the non-termination of further queries.

These points show that there are several potential ways to directly build on the work done as part of this thesis.
Beyond that, approaching the verification of \gls{prins} from another angle, for example modeling it for a completely different tool may yield additional insights.
Rather than a conclusive analysis, this thesis should instead be seen as a first step in verifying the security of \gls{prins}.
Ideally, it can contribute to a greater awareness of this protocol within the security research community and by that, the security of the 5G system as a whole.