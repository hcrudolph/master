The fifth generation of mobile networks, 5G, is expected to enable ubiquitous broadband access and a range of novel use cases, such as the massive Internet of Things, lifeline communication, and e-health services (see \cite{ngmn5Gwhite}).
Naturally, as our society grows to be more connected and the applications of this technology become increasingly critical, ensuring its safety and security is vital not only to prevent harm to digital assets but to human lifes as well.
Doing so requires comprehensive and continuous efforts that include --among others-- secure technical specifications, reliable implementations, as well as secure operations and management.
This document solely focussed on one part of the 5G specification produced by the \gls{3gpp}, an international consortium of \glspl{mno}, network equipment vendors, regulators, and other stakeholders.

A fundamental requirement for global connectivity in the \gls{3gpp} ecosystem is communication between different \glspl{plmn}.
Whether a subscriber visits a foreign network, commonly known as \textit{roaming}, or locally consumes services that involve parties in different mobile networks (e.g. international calls), a \gls{plmn} needs to exchange data with its peers.
This communication is routed through a global network called \gls{ipx}, separate from the Internet.
Originally thought of as a private communication channel of the mobile operator community, it is known for being a security risk that operators are inevitably exposed to due to a number of shortcomings.
Firstly, it lacks proper access control to restrict its members to genuine mobile network operators only, as these companies are regularly reselling their connectivity as part of partnerships or even end-user offerings.
Secondly, the fact that some of the protocols used on the \gls{ipx} network do not offer basic security features such as authentication or confidentiality makes this interface particularly prone to malicious traffic.
Thirdly, the network infrastructure itself is operated by third-party companies which have full access to signalling traffic that passes through their peering points.
Although the associated risks can be minimized by appropriate security controls, the inherent weaknesses of inter-operator signaling renders it one of the major fraud risks for mobile network operators (\cite{sahin2017sok}).

\gls{3gpp} attempts to improve the security of inter-operator signaling in its technical specifications of the 5G system.
As part of an overall updated protocol stack, release 15 and beyond include a new security protocol for the so-called N32 interface (see \cite{3gpp.33.501}).
The \gls{prins} is purposefully designed by \gls{3gpp}'s Security Working Group to fit the specific requirements of inter-operator signaling, described in detail in clause \ref{sec:n32}.
Although parts of this protocol build on established technologies such as \gls{tls} and \gls{jose}, secure protocol design remains a non-trivial task in which defects are difficult to identify by manual analysis.
A prominent negative example is the protocol proposed by Needham and Schroeder in 1978 (\cite{needham1978using}).
Initially considered secure, it was proven to be flawed 17 years after its inception (\cite{lowe1996breaking}).

One tool that can facilitate the design of secure systems and algorithms is formal verification -- establishing correctness proofs with respect to its requirements by means of formal methods of mathematics.
Since this process is based on a formalized model of the system under design it is also known as ,,model checking''.
According to Alur ,,a designer first constructs a model, with mathematically precise semantics [...] and performs extensive analysis with respect to correctness requirements'' (\cite{alur2011formal}).
In combination with computer-aided verification tools, that are reviewed in section \ref{sec:formal}, this approach can help to discover logical flaws that would remain undetected with other static testing approaches -- the above-mentioned Needham-Schroeder protocol only being one example.
It should be noted that formal verification can only ever produce proofs about the model it is applied to.
Proving facts about a model that does not accurately reflect the actual system yields limited insights.

In light of 5G introducing a new inter-operator signaling protocol, the question arises whether \gls{prins} succeeds to cater to the particular requirements of the N32 interface while simultaneously delivering on its desired security properties.
