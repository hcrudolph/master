Initially, section \ref{sec:n32} examines the relevant \gls{3gpp} specifications in detail to establish a firm understanding of all protocol components and messages flows, involved participants, and desired security properties of \gls{prins}.
Since the protocol features some very distinct requirements, the assumed trust model is specifically mentioned.
This first step serves as the basis for transposing the specification into a formal model later on.
Secondly, \ref{sec:formal} goes on to describe some of the theoretical preliminaries of symbolic model checking for the purposes of verifying security protocols.
It is described what kind of security properties can be verified using this approach and what the limitations are.
Subsequently, the two popular model checking tools \textsc{Tamarin Prover} and \textsc{ProVerif} , in order to determine the one that best covers the properties of the protocol under test.
Aside from the security requirements to be verified and a particular tool's feature set --which on their own may not conclusively answer this question-- additional aspects, such as the degree of automation, are also considered.
The choice of software further determines the model created in the next step.
Hence, section \ref{sec:model} provides a detailed account of the process of modeling \gls{prins} in a way that is suitable to be verified by the chosen tool.
Since \gls{3gpp} documents are not written in a strictly formal manner but rather in prose, the protocol specification may prove not to be perfectly complete -- a point of critique \cite{basin2018model} highlight with respect to the 5G \gls{aka} protocol.
In such cases, explicit assumptions shall be made to be able to complete the computer-aided verification.

Afterwards, section \ref{sec:analysis} discusses the issues encountered during execution of the model checker as well as the information obtained as a result.
Section \ref{sec:implications} puts these findings into context and outlines the practical security implications on the \gls{prins} protocol.
Where required, recommendations for improving the protocol's specificatiom are provided.
