This section describes the methods employed in order to answer the previously defined research questions.
As such, it also outlines the high-level structure of the remainder of this thesis.

Initially, section \ref{sec:n32} examines the relevant \gls{3gpp} specifications in detail to understand all involved parties, security requirements, as well as message contents and sequences of the \gls{prins} protocol.
This serves as the basis for transposing the specification into a formal model later on.
Subsequently, a comparison between popular model checking tools is performed in section \ref{sec:formal}, in order to determine the one that best covers the properties of the protocol under test.
Aside from the security requirements to be verified and a particular tool's feature set --which on their own may not conclusively answer this question-- additional aspects, such as the degree of automation, are also considered.
The choice to utilize a certain software also determines the model that is created in a next step.
Therefore, section \ref{sec:model} provides a detailed account of the process of modelling \gls{prins} in a way that is suitable to be verified by the chosen tool.
Since \gls{3gpp} documents are not written in a strictly formal manner, but rather in prose, the protocol specification may prove not to be perfectly complete -- a point of critique \cite{basin2018model} highlight with respect to the 5G \gls{aka} protocol.
In such cases, explicit assumptions shall be made to be able to complete a computer-aided verification.

Afterwards, section \ref{sec:analysis} discusses the issues encountered during execution of the model checker as well as the information obtained as a result.
Section \ref{sec:implications} puts these findings into context and outlines the practical security implications on the \gls{prins} protocol.
Where required, recommendations for improving the protocol's specificatiom are provided.
