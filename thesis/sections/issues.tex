\subsection{Deliberate Abstractions}
\label{ssec:abstractions}

The \textsc{ProVerif} model of \gls{prins} described in the previous section necessarily abstracts away a number of details that have to be considered by real-world implementations.
In the following, these intentional abstractions and their implications on the verification of security properties are discussed.

\subsubsection{Perfectly Secure TLS \& Key Derivation}

As outlined before, this analysis solely focusses on the novel parts of the \gls{prins} protocol defined by \gls{3gpp}.
\gls{tls} is assumed to be perfectly secure and thus, is not modelled in detail.
Instead, \gls{tls}-based connections are represented by secure (i.e. confidential and integrity protected) channels shown in listing \ref{lst:channels}.
Since mutual authentication in N32-c communication crucially depends on the security of the transport layer, this model is not able to effectively validate this property of N32-c.

Another protocol aspect closely coupled with the \gls{tls} protocol is derivation of cryptographic key material during the N32-c handshake procedure.
According to the specification ,,two \glspl{sepp} shall export keying material from the \gls{tls} session established between them using the \gls{tls} export function'' (\cite{3gpp.33.501}, p. 132).
This export function produces a pseudorandom bitstring that is used to derive the various N32-f session keys outlined in section \ref{sec:n32}.
The model's N32-c channel being inherently secure by assumption, it is not possible to model a similar function producing fresh key material based on a shared, authenticated session.
This issue is being addressed, by the private constructor {\sffamily deriveMasterKey}, consuming the N32-f context ID.
This input parameter ensures that the resulting master key is unique to a specific N32-f session.
The attribute {\sffamily private} guarantees that this function is not be accessible to adversaries, so that even with possession of the N32-f context ID, it is not possible to derive the same cryptographic key.
This kind of substitution for the \gls{tls} export function is flawed in two ways. Firstly, it relies on the secrecy of the derivation function iteslf, rather than the secrecy of the input parameter.
Secondly, the attribute {\sffamily private} still allows the function to be used by all legitimate protocol participants.
In the created model, this also includes the \gls{ipx} processes.
However, these points do not impact verification of the protocol, as private \textsc{ProVerif} functions cannot be leaked and the honest \gls{ipx} processes do not make use of the function either.

\subsubsection{Protection Policies}

\gls{prins} protection policies control what data-types are to be ciphered before being sent over the N32-f channel and which authenticated modifications may be performed on them.
Both types of protection policies are abstracted to single bitstrings in order to simplify related operations.

The data-type encryption policy plays a key role in rewriting \gls{http} messages received from \glspl{nf} in the \gls{sepp}'s own \gls{plmn}.
In the model, confidential and non-confidential message parts are extracted from an \gls{http} message body using the constructors {\sffamily getConf} and {\sffamily getNonconf}, displayed in listing \ref{lst:n32f-rewriting}.
Using an encryption policy as a second argument, they enable honest participants to identify those information elements only requiring integrity protection and those additionally requiring confidentialiy.
Note, that since the model does not consider protection policies confidential information and makes them publicly available, a flawed representation is avoided in which the adversary would not be able to retrieve sensitive information from the body of an \gls{http} message without the related encryption policy.

Before the receiving \gls{sepp} compiles the resulting \gls{http}message, it uses the {\sffamily policyValidation} deconstructor to verify whether the sending peer actually applied the previously agreed encryption policy or whether sensitive elements are actually contained in non-confidential parts of the N32-f message.
This is in addition to the equality check of both locally stored and received encryption policies during the initial N32-c handshake.

Finally, the the constructor {\sffamily applyPatches} is used by the receiving side to combine the individual message parts back to an \gls{http} message under consideration of the intermediary changes.

\begin{lstlisting}[caption={N32-f message rewriting},label={lst:n32f-rewriting},firstnumber=105]
(* data-type encryption policy application *)
fun getConf(bitstring, bitstring): bitstring.
fun getNonconf(bitstring, bitstring): bitstring.
reduc forall m: bitstring, p: bitstring;
    policyValidation(getNonconf(m, p), p) = true.
\end{lstlisting}

The bitstrings modelling the modification policies represent both the policies defined by either of the \gls{sepp} operators as well as the actual modifications written by intermediary \gls{ipx} providers.
This allows a simple comparison of the operation contained inside the \gls{json} patch and the related policy itself to determine whether or not a given message modification is legitimate.
This does not impact the verification result, as it is logically the same as a policy containing exactly one entry.
If the protocol behaves as intended using this examplary modification policy, it behaves similarly for more complex examples.

\subsubsection{Message Routing}

As mentioned in the previous section, inter-\gls{plmn} message routing, which would utilize \gls{ip} routing in real-world implementations, is simplified greatly by use of dedicated N32-c and N32-f channels.
Processes of sending \glspl{sepp} specify their communication peer by its address in outgouing messages.
This information is provided to avoid trivial loops between sending and receiving processes of the same \gls{sepp}:
Without any attribute identifying either part of the communication, an outgoing message could be consumed by an instance of the same \gls{sepp}'s receiving process without noticing the message is in fact its own.
Hence, in the model receiving both \glspl{sepp} verify whether the identity specified in incoming messages matches their own and only proceed with processing the message if this condition is met.
Note that this does not preclude the verification of potential authentication flaws -- an adversary is, of course, still able to spoof the \gls{sepp} address value.
The check merely prevents the confusion of messages by honest \gls{sepp} processes.

Beyond that, the model does not consider any information identifying the source of a message.
That is, at least on the N32-c channel, which is considered mutually authenticated as explained previously.
As far as N32-f communication is concerned, the receiving \gls{sepp} is able to correlate an incoming message by its N32-f context ID to the locally stored session parameters, such as session keys.
By validating the \gls{aad} contained in the message, the \gls{sepp} is able to implicitly verify the authenticity of the original message contents.

\subsubsection{IPX Provider Internals}

Similarly to message routing from and to communicating \glspl{sepp}, the message handling by \gls{ipx} providers is reduced to the bare minimum.
Each \gls{ipx} process utilizes two distinct N32-f channels, one towards its directly connected \gls{sepp} and another towards the second \gls{ipx} provider.
Every message received on one of these interfaces is forwarded on the other one after internal processing.

Modifications are signed using raw private keys, one of two options specified by the \gls{3gpp} specification for this purpose.
The alternative, i.e. a digital certificate issued by the operator of the directly connected \gls{sepp}, as well as the possibility for \gls{ipx} providers to use multiple cryptographic identities is not considered in the model.
Since there is no fundamental difference between the use of one or more keys and the asymmetric cryptography used in digital certificates is the same as for raw private/public key pairs, this abstraction does not negatively impact the verification result.

\subsubsection{Parameter Renegotiation}

Renegotiation of session parameters is left out of scope of this work entirely.
Such procedure is required when existing session keys reach the end of their lifetime or any other preconfigured session parameter, such as the cipher suite to be used, changes.
Firstly, at the time of this writing the \gls{3gpp} specifications \gls{ts} 33.501 and \gls{ts} 29.573 do not clearly stipulate how a \gls{sepp} is to perform a renegotiation.
Secondly, the triggers that would necessitate this procedure, such as the timeout of cryptographic keys are not considered in the model either, for reasons described in the following section.

\subsection{Model Shortcomings}

\subsubsection{Invalidation of N32-f Keys \& Contexts}

Formal verification in general provides very limited support for expressing temporal relations.
While \textsc{ProVerif} allows for the verification of correspondence assertions, ensuring that certain events do or do not happen after other events, it does not allow to convey precise timings.
Hence, the model does not consider the limited lifetime of cryptographic keys associated with a N32-f context.

Furthermore, real-world protocol implementations relying on incremental counters, such as the message IDs used in N32-f communication, are forced to handle limits of integer values to avoid overflows.
At some point during protocol execution, session parameters have to be refreshed in order to prevent counter wrap around.
\textsc{ProVerif} does offer support for natural numbers, however, the concept of integer overflows is not reproduceble.

Lastly, \textsc{ProVerif} tables, which represent a convenient way of modelling local \gls{sepp} storage as shown in the previous sections, do not offer the possibility of deleting records.
While real-world implementations would invalidate N32-f contexts at certain events or specific points in time, the model is not able to reproduce this behavior.

\subsubsection{Behavior of N32-c Channels}

According to the specification, the N32-c is unidirectional with regards to the initiation of a message exchange.
While a peer may respond on the same logical channel, only the initiating \gls{sepp} can actively issue requests.
This is due to the nature of the underlying \gls{http} protocol, which follows a client-server model.
The initiating \gls{sepp} takes on the role of the client, the responding \gls{sepp} that of the server.
In order to enable the responding end to initiate requests as well, for example to perform error signaling, a second N32-c connection in the inverse direction is established at the end of the initial handshake.

\textsc{ProVerif} channels, no matter if private or not, are always bidirectional.
While this fails to restrict messages being sent in either direction, one can work around this limitation during the modelling phase by letting sending and responding processes of both \glspl{sepp} use separate N32-c channels.
Since private channels, unlike public ones, cannot be created ad hoc during the protocol execution, both N32-c channels have to be declared in advance.

An additional characteristic of private channels that does not mirror a real-world \gls{tls} connection is that every information sent over them can be received by any of the listening participants at any time.
As the user manual explains, ,,ProVerif just computes a set of messages sent on a private channel, and considers that any input on that private channel can receive any of these messages (independently of the order in which
they are sent)'' (\cite{blanchet2020proverif}, p. 119).