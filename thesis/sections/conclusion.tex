Looking back at section~\ref{sec:goal}, the work done as part of this thesis successfully manages to address the objectives initially outlined.
The three research questions can be answered as follows.\bigskip

\noindent
(1) Is it possible to comprehensively describe the 5G inter-operator signaling protocol with existing modeling techniques?

Yes, with a number of abstractions.
The high-level comparison between \textsc{Tamarin} and \textsc{ProVerif} shows that both tools should in theory be able to express and validate important aspects of the \gls{prins} protocol.
Only a few useful modeling constructs in \textsc{ProVerif}, such as tables, synchronization, and better support for observational equivalence proofs, favor this tool.

The created model successfully incorporates all key characteristics of the \gls{prins} protocol, including multi-channel communication, separation of protocol phases, multiple active participants, intermediary message modifications, and all necessary cryptographic operations (i.e. signatures, \glspl{mac}, and \gls{aead} encrytion).
Minor aspects are abstracted away, such as the detailed structure of protection policies, the \gls{tls} key export, and the invalidation of N32-f sessions as described in section~\ref{sec:issues}.
However, given their impact on the security properties to be verified, these do not have an impact on the verification result.

That said, a degree of uncertainty about the model's correctness persists, due to the problems encoutered during verification.
Although all debug queries succeed and thereby greatly increase the confidence in the model, for the reasons layed out in section~\ref{sec:lessons}, there is still a chance \textsc{ProVerif}'s execution actually diverges from the intended protocol behavior.\bigskip

\noindent
(2) Does modeling \gls{prins} in a formal grammar reveal any inconsistencies that should be addressed by the \gls{3gpp} specification?

Yes, the previous section contains a summary of several areas that were either ambiguous or completely missing from the specifications \gls{ts} 33.501 and \gls{ts} 29.573.
Some of these should be amendable by ensuring a uniform naming scheme between the two documents or minor additions of missing procedures.
Other aspects, such as a clearly defined threat model and security requirements, are more fundamental in nature and likely require comprehensive revision of the specification.
While the authors of the main security specification \gls{ts} 33.501 might have a comprehensive view of these aspects, they are not explicitly captured in the document or any related \gls{3gpp} study documents.
Not only does this impede a clear understanding for implementers and mobile operators of what the protocol is supposed to deliver and what might be necessary to address by additional, non-standard controls.
But also, it inhibits security research to properly verify these protocols against their intended properties.

The author argues that, given the importance of secure and reliable communication for some of the mission critical use cases 5G is supposed to address, specifying a concrete threat model and security requirements derived from it is indispensable.
More so than a formal verification, which proved rather fragile in practice, a precise definition of the requirements would greatly facilitate the development of a secure protocol.\bigskip

\noindent
(3) Based on the model created as part of this work, can current formal verification tools proof or disproof particular security requirements of the \gls{prins} protocol?

Yes, partially.
Section \ref{sec:verification} shows that confidentiality protection for all sensitive information is indeed ensured if the protocol is implemented according to the specification.
Note, that since the verification is performed in the symbolic domain, cryptographic flaws such as nonce reuse resulting in a breach of confidentiality with \gls{aead} ciphers, cannot be proven.
Moreover, message modifications by unknown \gls{ipx} providers or modifications not captured in the modification policy are never accepted by the receiving party.

Properties that were disproven include the rejection of replayed message modifications, detection of the removal of previously added message modifications, as well as the integrity protection of \gls{json} patches.
While the respective queries leave some margin of error for potential false attacks to be found by \textsc{ProVerif}, as outlined in section~\ref{sec:verification}, there is no way to tell without the discovered attack traces available.
What is more, the authentication of N32-f messages itself could not be proven, as it resulted in the non-termination of \textsc{ProVerif}.

Based on these results, it would be premature to speak of a vulnerability or weakness in the \gls{prins} protocol itself.
Instead, further work is needed in order to validate the created model, refine the verification performed as part of this thesis, and substantiate the findings.
A few approaches that may guide future efforts are outlined in the last section.