Based on the lessons learned during by modelling and formally verifying of the \gls{prins} protocol, a number of gaps and inconsistencies in the \gls{3gpp} specification have been identified.
When addressed, these points can facilitate a more coherent specification, less ambiguity during implementation, and by that a more sound protocol as a whole.

On of the first observations when analyzing the specification \gls{ts} 33.501 is the absence of an explicitly defined threat model and security requirements derived from it.
Previous research on another component of the 5G system has highlighted this as an issue (see \cite{basin2018model}) and it holds true for the decription of the inter-operator protocol as well.
In fact, merely half of the security properties outlined in section \ref{ssec:prins-sequence} are explicitly spelled out in the \gls{3gpp} document.
The lack of a clear description of what properties the protocol is supposed to achieve not only impedes the creation of related queries for a formal verification, it is also likely to cause issues during implementation.

Additionally, \gls{3gpp} documents \gls{ts} 33.501 and \gls{ts} 29.573 describing the \gls{prins} protocol in different levels of detail are misaligned in several locations.

\begin{enumerate}[label=--]
\item While  \gls{ts} 33.501 only refers to an {\sffamily authorized IPX ID} (\cite{3gpp.33.501}, p. 139), \gls{ts} 29.573 mentions both a {\sffamily authorized IPX ID} and a {\sffamily next hop ID} in multiple locations (\cite{3gpp.29.573}, p. 65-66). Based on their location in the N32-f message, it can to be assumed that these paramters actually refer to the same information.

\item \gls{ts} 33.501 specifies the {\sffamily PLMN ID} is supposed to be validated by the receiving \gls{sepp} (\cite{3gpp.33.501}, p. 143). However, it does not detail where in the N32-f message the parameter is transferred. \gls{ts} 29.573 cites an authorization header containing this information (\cite{3gpp.29.573}, p. 19).

\item Although \gls{ts} 33.501 defines a validity parameter as part of the N32-f context information (\cite{3gpp.33.501}, p. 134), it does not clearly specify to what exactly this parameter applies -- the whole context including protection policies or just the cryptographic material. Neither \gls{ts} 33.501 nor \gls{ts} 29.573 specify the intended behavior of a \gls{sepp} once this validity has expired.
\end{enumerate}

