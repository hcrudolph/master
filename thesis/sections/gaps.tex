Based on the lessons learned during modeling and formally verifying of the \gls{prins} protocol, a number of gaps and inconsistencies in the \gls{3gpp} specification have been identified.
When addressed, these points can facilitate a more coherent specification, a reduction of potential ambiguity during implementation, and by that an overall sounder protocol.

One of the first observations when analyzing the specification \gls{ts} 33.501 is the absence of an explicitly defined threat model and security requirements derived from it.
Previous research on another component of the 5G system highlighted this as an issue (see \cite{basin2018model}) and it holds true for the decription of the inter-operator protocol as well.
In fact, merely half of the security properties outlined in section~\ref{ssec:prins-sequence} are explicitly spelled out in \gls{3gpp}'s document.
The lack of a clear description of what properties the protocol is supposed to achieve not only impedes the creation of related queries for a formal verification, it is also likely to cause issues during implementation.

Further, the specification makes assumptions about certain information being agreed and correctly configured between protocol participants prior to session establishment without detailing the intended behavior if any of these assumptions is not met.
As highlighted in section~\ref{ssec:prins-structure}, it fails to describe if and how the N32-c session establishment is to proceed if either of the data-type encryption policies violates the protection needs of the information elements to be mandatorily encrypted.

Additionally, \gls{3gpp} documents \gls{ts} 33.501 and \gls{ts} 29.573 describing the \gls{prins} protocol in different levels of detail are misaligned in several locations.

\begin{enumerate}[label=--]
\item While \gls{ts} 33.501 only refers to an {\sffamily authorized IPX ID} (\cite{3gpp.33.501}, p.~139), \gls{ts} 29.573 mentions both a {\sffamily authorized IPX ID} and a {\sffamily next hop ID} in multiple locations (\cite{3gpp.29.573}, pp.~65-66). Based on their placement within the N32-f message, it can be assumed that these parameters actually refer to the same information, but this connection should be made more clear.

\item \gls{ts} 33.501 specifies that the {\sffamily PLMN ID} is supposed to be validated by the receiving \gls{sepp} (\cite{3gpp.33.501}, p.~143). However, it does not detail where in the N32-f message the parameter is transferred. \gls{ts} 29.573 cites an authorization header containing this information (\cite{3gpp.29.573}, p.~19). Again, explicitly spelling out this relation would improve the coherence of the two specifications.

\item Although \gls{ts} 33.501 defines a validity parameter as part of the N32-f context information (\cite{3gpp.33.501}, p.~134), it does not clearly specify to what exactly this parameter applies -- the whole context including protection policies or just the cryptographic material.

\item Neither \gls{ts} 33.501 nor \gls{ts} 29.573 specify the intended behavior of a \gls{sepp} once the above-mentioned validity has expired. While the latter contains a description of a N32-f Context Termination Procedure (\cite{3gpp.29.573}, p.~17), it is unclear whether that is automatically triggered when the end of the context validity is reached or if some parameter renegotiation should be performed instead.
\end{enumerate}

While \gls{sepp} vendors would surely be able find a solution to these inconsistencies during implementation, it is not guaranteed that other system suppliers would opt for the same behavior in their components.
Given the purpose of standardization is ensuring compatibility between system components, it seems as though the specification of \gls{prins} is still too vague to ensure this.