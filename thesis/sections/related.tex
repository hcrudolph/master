The idea of proving correctness of concurrent systems, such as network protocols, based on a formal representation exists for several decades already.
\cite{basin2018model} provide an extensive historical account of model checking security protocols, streching back to the 1980s.
This section initially outlines a brief overview of foundational theories and key contributions in this active field of reasearch.
Subsequently, recent work directly related to the application of formal verification to mobile network security is discussed.
Lastly, model cheking software commonly used in recent years are hightlighted, which serves a basis for choosing a particular tool in section \ref{sec:formal}.

The term ,,model checking'' as it is used today goes back to Clarke and Emerson, who are the first to propose the computer-aided verification of finite-state systems expressed in \gls{gls:ptl}.
In contrast to previous efforts, they model the system flow graph as a finite structure and describe ,,an efficient algorithm [...] to decide whether a given finite structure is a model of a particular formula'' (\cite{clarke1981design}, 2).
Even though this particular work is used to reason about a single concurrent program, the authors already hint to the possible extension of their approach to ,,network communication protocols''.
\cite{dolev1983security} consitutes another landmark paper in terms of formally verifying security protocols.
Their contribution is to model a protocol ,,as a machine consisting of an arbitrary number of honest agents executing the protocol, in which all messages sent are interceptedby the adversary [...], all messages received are sent by the adversary'' (\cite{basin2018model}, 6).
A foundational concept for many model checkers to this day, it is shown in section \ref{sec:n32} how this assumption is an excellent fit for the use case of inter-operator signaling.

The Dolev-Yao model further represents all messages in abstract mathematical terms and assumes the underlying cryptography to be perfect -- an approach also known as formal verification in the symbolic model or simply \textit{symbolic model checking}.
This stands in contrast to \textit{computational model checking} in which ,,messages are bitstrings, and the adversary is a probabilistic polynomial-time Turing machine'' (\cite{blanchet2008computationally}, 1).
Although successfully used for verifying security properties of several popular security protocols (see~\cite{BlanchetJaggardScedrovTsayAsiaCCS08}, \cite{CadeBlanchetJoWUA13}, \cite{LippBlanchetBhargavanEuroSP19}), one of the basic assumptions of the thesis is perfect cryptography (see section~\ref{sec:goal}).
Therefore, it is only mentioned for informational purposes and shall not be pursued further.

In recent years, \gls{3gpp} specifications clearly capture the interest of the research community.
This does not just apply to the recently finalized 5G standard and related protocols, but also to earlier generations (see~\cite{alt2016cryptographic}, \cite{lee2014anonymity}).
However, these works mostly focus on the \gls{aka} protocol.

\cite{basin2018formal}, \cite{dehnel2018security}, \cite{borgaonkar2019new}, \cite{lanzenberger2017formal}

\cite{o2017mobile}

tools
\cite{cremers2008scyther} \cite{scyther}
\cite{meier2013tamarin} \cite{tamarin}
\cite{blanchet2016modeling} \cite{proverif}