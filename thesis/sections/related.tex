The idea of proving correctness of concurrent systems, such as network protocols, based on a formal representation exists for several decades already.
\cite{basin2018model} provide an extensive description of the historical developments of model checking security protocols, streching back to the 1980s.
Based on their work, this section provides a brief overview of some foundational theories and key contributions in this active field of reasearch.
Additionally, recent work directly related to the application of formal verification to mobile network security is highlighted.

The term ,,model checking'' as it is used today goes back to \cite{clarke1981design}, who are the first to propose the computer-aided verification of finite-state systems expressed in \gls{gls:ptl}.
In contrast to previous efforts, they model the system flow graph as finite structure and describe ,,an efficient algorithm [...] to decide whether a given finite structure is a model of a particular formula''.
Even though this particular work is used to reason about a single concurrent program, the authors already hint to the possible extension of their approach to ,,network communication protocols''.
\cite{dolev1983security} where the first ones to apply this technique to security protocols.

things to cover
symbolic/computational approach
protocol model/network model

tools
Cremers - The Scyther Tool: Verification, falsification, and analysis of security protocols
Meier - The TAMARIN prover for the symbolic analysis of security protocols
Blanchet - Modeling and verifying security protocols with the applied pi calculus and ProVerif

5g papers
Basin - A formal analysis of 5G authentication
Cremers - A comprehensive symbolic analysis of TLS 1.3
Cremers - Authentication vulnerability in the most recent 5G AKA drafts
Borgaonkar - New Privacy Threat on 3G, 4G, and Upcoming 5G AKA Protocols
O’Hanlon - Mobile Subscriber WiFi Privacy
Lanzenberger - Formal Analysis of 5G Protocols
