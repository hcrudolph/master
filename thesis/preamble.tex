\usepackage{geometry}
\geometry{a4paper, top=20mm, left=25mm, right=25mm, bottom=20mm}
\usepackage[T1]{fontenc}
\usepackage[utf8]{inputenc}
\usepackage[justification=centering]{caption}
\usepackage{graphicx}
\usepackage{amsmath}
\usepackage{amssymb}
\usepackage{array}
\usepackage{colortbl}
\usepackage{rotating}
\usepackage{relsize}
\usepackage{wrapfig}
\usepackage{multirow}
\usepackage[title,titletoc]{appendix}
\usepackage[dvipsnames]{xcolor}
\usepackage[hidelinks,hyperfootnotes=false]{hyperref}
\usepackage[toc,acronym,nonumberlist,nogroupskip,nopostdot]{glossaries}
\usepackage{csquotes}
\usepackage[
    backend=biber,
    style=authoryear,
    sorting=nyt,
    citestyle=authoryear]{biblatex}
\addbibresource{lit.bib}
\addbibresource{lito.bib}
\usepackage[nottoc]{tocbibind}
\usepackage{imakeidx}
\usepackage{import}
\usepackage{listings}
\usepackage{enumitem}
\usepackage{setspace}
\usepackage{awesomebox}
\usepackage{afterpage}
\usepackage{soul}
\usepackage[titles]{tocloft}

\graphicspath {{figures/}}
\hyphenation{}

% custom commands

\newcommand{\blankpage}{
        \null
        \thispagestyle{empty}
        \addtocounter{page}{-1}
        \newpage
    }

\newcommand{\ltracerule}{
    \mathrel{\ooalign{\hss\cr\kern-.1ex$-$\hss\cr\kern1.2ex\hbox{[}}}
}

\newcommand{\rtracerule}{
    \mathrel{\ooalign{\hss\cr\kern0ex]\hss\cr\kern.2ex\hbox{$\rightarrow$}}}
}

\newcommand{\tracerule}[1]{
    \mathrel{\ooalign{\hss\cr\kern-.1ex$-$\hss\cr\kern1.2ex\hbox{[}}}{#1}\mathrel{\ooalign{\hss\cr\kern-.2ex]\hss\cr\kern0ex\hbox{$\rightarrow$}}}
}

\setlength{\aweboxvskip}{3mm}
\newcommand{\sgoal}[1]{
    \awesomebox[RoyalBlue]{2pt}{\faUnlock*}{RoyalBlue}{#1}
}

\newcommand{\asgoal}[1]{
    \awesomebox[Gray]{2pt}{\faUnlock*}{Gray}{#1}
}

\renewcommand{\baselinestretch}{1.15}

\newcommand*{\lstcomment}[1]{\hfill\makebox[8cm][l]{#1}}

\let\origthelstnumber\thelstnumber
\makeatletter
\newcommand*\Suppressnumber{%
  \lst@AddToHook{OnNewLine}{%
    \let\thelstnumber\relax%
%    \advance\c@lstnumber-\@ne\relax% Not really necessary
  }%
}

\newcommand\Reactivatenumber[1]{%
  \global\c@lstnumber#1%
  \global\advance\c@lstnumber\m@ne\relax%
  \lst@AddToHook{OnNewLine}{%
  \let\thelstnumber\origthelstnumber%
  }%
}
\makeatother


% count footnotes globally
\usepackage{chngcntr}
\counterwithout{footnote}{chapter}

% listing styles
% \lstset{
%     frame=shadowbox,
%     rulesepcolor=\color{black},
%     backgroundcolor=\color{white},      % choose the background color; you must add \usepackage{color} or \usepackage{xcolor}
%     basicstyle=\ttfamily\footnotesize,  % the size of the fonts that are used for the code
%     breakatwhitespace=false,            % sets if automatic breaks should only happen at whitespace
%     breaklines=true,                    % sets automatic line breaking
%     captionpos=b,                       % sets the caption-position to bottom
%     commentstyle=\color{gray},          % comment style
%     deletekeywords={...},               % if you want to delete keywords from the given language
%     escapeinside={(@*}{@*)},            % if you want to add LaTeX within your code
%     extendedchars=true,                 % lets you use non-ASCII characters; for 8-bits encodings only, does not work with UTF-8
%     keepspaces=true,                    % keeps spaces in text, useful for keeping indentation of code (possibly needs columns=flexible)
%     keywordstyle=\lst@ifdisplaystyle\color{blue}\fi,       % keyword style
%     language=Haskell,                   % the language of the code
%     otherkeywords={>>,>>=},             % if you want to add more keywords to the set
%     numbers=left,                       % where to put the line-numbers; possible values are (none, left, right)
%     numbersep=5pt,                      % how far the line-numbers are from the code
%     numberstyle=\tiny\color{gray},      % the style that is used for the line-numbers
%     showspaces=false,                   % show spaces everywhere adding particular underscores; it overrides 'showstringspaces'
%     showstringspaces=false,             % underline spaces within strings only
%     showtabs=false,                     % show tabs within strings adding particular underscores
%     stepnumber=1,                       % the step between two line-numbers. If it's 1, each line will be numbered
%     stringstyle=\color{blue},           % string literal style
%     tabsize=2,                          % sets default tabsize to 2 spaces
%     title=\lstname                      % show the filename of files included with \lstinputlisting; also try caption instead of title
% }

\newglossaryentry{gls:ptl}
{
    name={Propositional Temporal Logic},
    description={see: \gls{gls:ltl}}
}

\newglossaryentry{gls:ltl}
{
    name={Linear-time Temporal Logic},
    description={describes a system of rules and modalities referring to time that allows one to ,,describe and reason about how the truth values of assertions vary
with time'' (\cite{emerson1990temporal}, 1). Such system is linear in that every point in time has a defined successor.}
}

\newglossaryentry{gls:csec}
{
    name={Concrete Security},
    description={.}
}

\newglossaryentry{gls:2g}
{
    name={2G},
    description={also known as Global System for Mobile Communications (GSM) is the first global mobile communication standard. Developed by the European Telecommunications Standards Institute in 1991, it still in widespread use worldwide at the time of this writing.}
}

\newglossaryentry{gls:3g}
{
    name={3G},
    description={also known as Universal Mobile Telecommunication System (UMTS), is the term commonly used to describe 3GPP Releases 99 through 7 (i.e. 99, 4, 5, 6, 7). The first mobile standard to be developed by 3GPP, is reuses the \gls{gls:2g}/GSM core network paired with a redesigned \gls{gls:ran}.}
}

\newglossaryentry{gls:4g}
{
    name={4G},
    description={also known as Long Term Evolution (LTE), is the term commonly used to describe 3GPP Releases 8 through 14. It is the first global standard for purely packet-switched public mobile network.}
}

\newglossaryentry{gls:cn}
{
    name={Core Network},
    description={describes the part of a mobile network that fulfills central functionalities, such as subscriber authentication, mobility, etc. It is complementary to the Radio Access Network.}
}

\newglossaryentry{gls:ran}
{
    name={Radio Access Network},
    description={describes the part of a mobile network that transmit/receives radio signals to/from mobile handsets. It is complementary to the \gls{gls:cn}. In 5G, the RAN is comprised of base stations called gNB.}
}

\newglossaryentry{gls:CP}
{
    name={Signaling Data},
    description={describes protocol messages containing information used to manage the flow of user data.}
}

\newglossaryentry{gls:UP}
{
    name={User Data},
    description={describes protocol messages containing information that is directly created or consumed by the user.}
}

\newacronym{prins}{PRINS}{PRotocol for N32 INterconnect Security}
\newacronym{3gpp}{3GPP}{Third Generation Partnership Project}
\newacronym{plmn}{PLMN}{Public Land Mobile Network}
\newacronym{ipx}{IPX}{IP eXcange}
\newacronym{ss7}{SS7}{Signaling System 7}
\newacronym{tls}{TLS}{Transport Layer Security}
\newacronym{jose}{JOSE}{JSON Object Signing and Encryption}
\newacronym{jwe}{JWE}{JSON Web Encryption}
\newacronym{jws}{JWS}{JSON Web Signature}
\newacronym{jwk}{JWK}{JSON Web Key}
\newacronym{jwa}{JWA}{JSON Web Algorithms}

\makeglossaries
\makeindex

\glsadd{json}
