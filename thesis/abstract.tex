\begin{abstract}
    This thesis analyzes the formal security properties of the \gls{prins}, specified by the \gls{3gpp} to protect signaling traffic between 5G mobile networks.
    Being an emerging technology that is believed to enable a plethora of novel and potentially critical applications, it is mandatory that 5G communication be effectively protected by design.
    Since inter-operator signaling exhibits specific functional requirements that existing security protocols fail to meet, \gls{3gpp} defines this purpose-built protocol that has not undergone thorough analysis by the broader research community at the time of this writing.

    Formal methods have successfully been used to validate and improve security protocols in the past.
    The nature of this approach to verifying a system's correctness makes it particularly suited to detect logical flaws in the design.
    Semi-automated tools for formal verification further aid the discovery of issues that would be non-trivial to spot by manual analysis.
    We assess the \gls{prins} specification in detail and create a model for the popular model checker \textsc{Tamarin}.
    
    By modeling the \gls{prins} protocol and formally verifying it against its intended security properties, we show that the \gls{3gpp} specification contains several inconsistencies and a lack of clarity about what the protocol is supposed to achieve.
    We argue that this ambiquity can lead to unreliable and therefore insecure implementations.
    Based on our findings, we suggest a number of improvements that can help make the specification more explicit and easier to understand.
\end{abstract}

\glsreset{prins}
\glsreset{3gpp}