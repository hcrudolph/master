\newglossaryentry{gls:ltl}
{
    name={Linear-time Temporal Logic},
    description={describes a system of rules and modalities referring to time that allows one to ,,describe and reason about how the truth values of assertions vary
with time'' (\cite{emerson1990temporal}, p. 1). Such system is linear in that every point in time has a defined successor.}
}

\newglossaryentry{gls:2g}
{
    name={2G},
    description={is an acronym for the second generation of cellular mobile networks. \gls{gsm}, a 2G-compliant standard, is commonly used synonymously due to its widespread implementation. Developed by the European Telecommunications Standards Institute in 1991, \gls{gsm} is still in use in many countries around the world at the time of this writing.}
}

\newglossaryentry{gls:3g}
{
    name={3G},
    description={is an acronym for the third generation of cellular mobile networks. \gls{umts}, a 3G-compliant standard, is commonly used to describe 3GPP Releases 99 through 7 (i.e. 99, 4, 5, 6, 7). The first mobile standard to be developed by 3GPP, is reuses the \gls{gls:2g}/GSM core network paired with a redesigned \gls{gls:ran}.}
}

\newglossaryentry{gls:4g}
{
    name={4G},
    description={is an acronym for the fourth generation of cellular mobile networks. \gls{lte}, a 4G-compliant standard, is commonly used to describe 3GPP Releases 8 through 14. It is the first global standard for purely packet-switched mobile network.}
}

\newglossaryentry{gls:5g}
{
    name={5G},
    description={is the official term used by 3GPP to describe its specification Releases 15 and beyond. An evolutionary change on top of 4G, it features a greater functional decoupling within the core network and an updated protocol stack.}
}

\newglossaryentry{gls:cn}
{
    name={Core Network},
    description={describes the part of a mobile network that host central functionalities, such as subscriber authentication, and mobility management. It is complementary to the \gls{gls:ran}.}
}

\newglossaryentry{gls:ran}
{
    name={Radio Access Network},
    description={describes the part of a mobile network that transmit/receives radio signals to/from mobile handsets. It is complementary to the \gls{gls:cn}. In 5G, the RAN is comprised of multiple base stations called gNodeB (gNB).}
}

\newglossaryentry{gls:cp}
{
    name={Control Plane},
    description={also known as Signaling Plane, describes protocol messages used to control the mobile network itself, incl. subscriber authentication, message routing, and session management.}
}

\newglossaryentry{gls:up}
{
    name={User Plane},
    description={describes protocol messages that are directly created or consumed by the user/user equipment and are only transported by the mobile network.}
}