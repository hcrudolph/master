\documentclass{beamer}
\usepackage[utf8]{inputenc}
\usetheme{Ilmenau}
\setbeamercovered{transparent}
\setbeamertemplate{navigation symbols}{} %remove navigation symbols

% Title Page Information
\title{Formal Verification of\\ The 5G Inter-Operator Signaling Protocol}
\subtitle{Master Thesis}
\author{Hans Christian Rudolph}
\institute{Hochschule Wismar}
\date{11. September 2020}
\logo{\includegraphics[height=0.5cm]{wings-logo.png}}

\begin{document}

% Title Page
\frame{\titlepage}

% Table of Contents
\begin{frame}
    \frametitle{Agenda}
    \tableofcontents
\end{frame}

% Problemstellung \& Kontext
\section{Problemstellung \& Kontext}

\begin{frame}
    \frametitle{Problemstellung \& Kontext}
    \framesubtitle{5G Inter-Operator Signalisierung}

    \begin{itemize}
        \item<1-> 3GPP definiert neue Protokollsammlung für 5G Signalisierung
        \vspace*{2mm}
        \item<1-> Eigens spezifiziertes PRINS Protokoll verspricht
        \begin{itemize}
            \item<1-> Variables Sicherheitslevel basierend auf ,,Protection Policies''
            \item<1-> Nachrichtenänderungen durch autorisierte Vermittler (IPX)
            \item<1-> Nachvollziehbarkeit und Überprüfbarkeit jener Änderungen
        \end{itemize}
    \end{itemize}
    \vspace*{3mm}
    \begin{enumerate}
        \item<2-> Ist PRINS Verifikation mit existiernden Werkzeugen möglich?
        \item<3-> Zeigt die Modellierung Lücken in der Spezifikation auf?
        \item<4-> Kann die formale Verifikation logische Fehler identifizieren?
    \end{enumerate}
\end{frame}

\begin{frame}
    \frametitle{Problemstellung \& Kontext}
    \framesubtitle{PRotocol for INterconnect Security (PRINS)}

    \begin{figure}
    \centering
    \includegraphics[width=\linewidth]{n32-interface-short.pdf}
    \end{figure}

    \begin{itemize}
        \item<2-> \textbf{Security Edge Protection Proxy} zentale Sicherheitsfunktion
        \item<3-> \textbf{N32-c} Kontrollkanal: Sitzungssteuerung, Fehlersignalisierung
        \item<3-> \textbf{N32-f} Datenkanal: Übertragung von Nutzerdaten
        \item<4-> \textbf{Data-type Encryption Policy}: Vertraulichkeitsanforderungen
        \item<4-> \textbf{Modification Policy}: legitime Nachrichtenänderungen
    \end{itemize}
\end{frame}

\begin{frame}
    \frametitle{Problemstellung \& Kontext}
    \framesubtitle{Formale Verifikation von Sicherheitsprotokollen}

    \begin{itemize}
        \item<1-> Grundlage formaler Verifikation ist temporale Logik
        \vspace*{2mm}
        \item<2-> Überprüfung eines Protokollmodells gegen logische Formeln
        \vspace*{2mm}
        \item<3-> \textbf{Model Checker} überprüfen Protokollzustände auf Erreichbarkeit, modale Zusammenhänge, etc.
        \vspace*{2mm}
        \item<4-> Symbolische Analyse kann allein logische Fehler identifizieren, keine kryptografischen Schwächen
        \vspace*{2mm}
        \item<5-> \textbf{Tamarin} und \textbf{ProVerif} sind populäre Programme zur symbolischen Modellprüfung
    \end{itemize}
\end{frame}

% PRINS Modellierung
\section{PRINS Modellierung}

\begin{frame}
    \frametitle{PRINS Modellierung}
    \framesubtitle{Überblick}

    \begin{itemize}
        \item<1-> ProVerif bietet mehrere nützliche Sprachkonzepte für PRINS (Synchronisation, private Tabellen, Ununterscheidbarkeit)
        \vspace*{2mm}
        \item<2-> Vier Protokollteilnehmer: SEPP A, SEPP B, IPX A, IPX B
        \vspace*{2mm}
        \item<3-> Angreifer implizit simuliert sowie ausgewählte Protokoll- verletzungen durch maliziöse IPX Provider nachgestellt
        \vspace*{2mm}
        \item<4-> Modellierte Prozesse:
        \begin{itemize}
            \item<4-> N32-c Sitzungsaufbau und Fehlersignalisierung
            \item<4-> N32-f Nachrichtenversand und -empfang
            \item<4-> IPX Nachrichtenänderungen
        \end{itemize}
    \end{itemize}
\end{frame}

\begin{frame}
    \frametitle{PRINS Modellierung}
    \framesubtitle{Abstraktionen \& Defizite}

    \begin{itemize}
        \item<1-> Nachrichtenrouting und IPX-Provider Interna
        \vspace*{2mm}
        \item<2-> Protection Policies als statische, eindimensionale Bitstrings
        \vspace*{2mm}
        \item<3-> Transport Layer Security (TLS) als sicherer Kanal
        \begin{itemize}
            \item[$\rightarrow$]<3-> Authentifizierung erfolgreich qua expliziter Annahme
            \item[$\rightarrow$]<3-> Generierung von Schlüsselmaterial mittels privater Funktion
        \end{itemize}
        \vspace*{2mm}
        \item<4-> Fehlende Invalidierung von Sitzungskontexten
        \begin{itemize}
            \item[$\rightarrow$]<4-> Unbegrenzte Validität sobald N32-c Sitzungsaufbau erfolgt
        \end{itemize}
    \end{itemize}
\end{frame}

% PRINS Verifikation
\section{PRINS Verifikation}

\begin{frame}
    \frametitle{PRINS Verifikation}
    \framesubtitle{Abfrageresultate}

    \begin{itemize}
        \item<1-> Sitzungsaufbau (Context ID, Keys) erfolgreich verifiziert
        \vspace*{2mm}
        \item<2-> Vertraulichkeitsanforderungen erfolgreich verifiziert
        \vspace*{2mm}
        \item<3-> Maliziöse IPX Änderungen teilweise erkannt
        \begin{itemize}
            \item[+]<3-> Ablehnen unbekannter IPX Provider
            \item[+]<3-> Ablehnen unbekannter Nachrichtenänderungen
            \item[--]<4-> Entfernen hinzugefügter Nachrichtenänderungen
            \item[--]<4-> Wiederholen vergangener Nachrichtenänderungen
        \end{itemize}
        \vspace*{2mm}
        \item<5-> Integrität und Authentizität von N32-f Nachrichten sowie IPX-Patches nicht erfolgreich verifiziert
    \end{itemize}
\end{frame}

\begin{frame}
    \frametitle{PRINS Verifikation}
    \framesubtitle{Herausforderungen}

    \begin{itemize}
        \item<1-> Eingeschränkte Möglichkeiten zur Identifikation von Fehlern
        \vspace*{2mm}
        \item<2-> Potenziell ausbleibende Terminierung von ProVerif
        \vspace*{2mm}
        \item<3-> Komplexes Rendering von Angriffssequenzen
        \vspace*{2mm}
        \item<4-> Fehlende Optimierung für Multiprozessorsysteme
    \end{itemize}
\end{frame}

% Fazit
\section{Fazit \& Ausblick}

\begin{frame}
    \frametitle{Fazit}

    \begin{itemize}
        \item[+]<1-> Individuelle Erreichbarkeit aller PRINS Protokollzustände
        \vspace*{2mm}
        \item[+]<1-> Erfolreiche Verifikation aller Vertraulichkeitsanforderungen
    \end{itemize}
    \vspace*{5mm}
    \begin{itemize}
        \item[--]<2-> ProVerif limitiert durch Graphviz Rendering-Algorithmus
        \vspace*{2mm}
        \item[--]<2-> Keine Möglichkeit Korrektheit des Modells zu überprüfen
    \end{itemize}
\end{frame}

\begin{frame}
    \frametitle{Ausblick}

    \begin{itemize}
        \item Verfeinern der Verifikation (Lösungsstrategien und Abfragen)
        \vspace*{2mm}
        \item Weitere Analyse der textbasierten Ausgabe
        \vspace*{2mm}
        \item Verbessern des Rendering-Algorithmus für gerichtete Graphen
    \end{itemize}
\end{frame}

\end{document}