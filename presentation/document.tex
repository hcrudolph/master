\documentclass{beamer}
\usepackage[utf8]{inputenc}
\usepackage{amssymb}
\usetheme{Ilmenau}
\setbeamercovered{transparent}
\setbeamertemplate{navigation symbols}{} %remove navigation symbols

% Title Page Information
\title{Formal Verification of\\ The 5G Inter-Operator Signaling Protocol}
\subtitle{Master Thesis}
\author{Hans Christian Rudolph}
\institute{Hochschule Wismar}
\date{11. September 2020}
\logo{\includegraphics[height=0.5cm]{wings-logo.png}}

\begin{document}

% Title Page
\frame{\titlepage}

% Table of Contents
\begin{frame}
    \frametitle{Agenda}
    \tableofcontents
\end{frame}

% Problemstellung und Kontext
\section{Problemstellung und Kontext}

\begin{frame}
    \frametitle{Problemstellung und Kontext}
    \framesubtitle{5G Inter-Operator Signalisierung}

    \begin{itemize}
        \item<1-> 3GPP definiert neue Protokollsammlung für 5G Signalisierung
        \vspace*{2mm}
        \item<2-> Eigens spezifiziertes PRINS Protokoll verspricht
        \begin{itemize}
            \item<2-> Variables Sicherheitslevel basierend auf ,,Protection Policies''
            \item<2-> Nachrichtenänderungen durch autorisierte Vermittler (IPX)
            \item<2-> Nachvollziehbarkeit und Überprüfbarkeit jener Änderungen
        \end{itemize}
    \end{itemize}
    \vspace*{3mm}
    \begin{enumerate}
        \item<3-> Ist PRINS Verifikation mit existierenden Werkzeugen möglich?
        \item<4-> Zeigt die Modellierung Lücken in der Spezifikation auf?
        \item<5-> Kann die formale Verifikation logische Fehler identifizieren?
    \end{enumerate}
\end{frame}

\begin{frame}
    \frametitle{Problemstellung und Kontext}
    \framesubtitle{PRotocol for INterconnect Security (PRINS)}

    \begin{figure}
    \centering
    \includegraphics[width=\linewidth]{n32-interface-short.pdf}
    \end{figure}

    \begin{itemize}
        \item<2-> \textbf{Security Edge Protection Proxy}: Sicherheitsfunktion
        \item<3-> \textbf{N32-c}: Kontrollkanal, Transport Layer Security
        \item<3-> \textbf{N32-f}: Datenkanal, Javascript Object Signing and Encryption
        \item<4-> \textbf{Data-type Encryption Policy}: Vertraulichkeitsanforderungen
        \item<4-> \textbf{Modification Policy}: legitime Nachrichtenänderungen
    \end{itemize}
\end{frame}

\begin{frame}
    \frametitle{Problemstellung und Kontext}
    \framesubtitle{Formale Verifikation von Sicherheitsprotokollen}

    \begin{itemize}
        \item<1-> Überprüfung eines Protokollmodells gegen logische Formeln
        \vspace*{2mm}
        \item<2-> \textbf{Model Checker} überprüfen Protokollzustände auf Erreichbarkeit und temporale Zusammenhänge
        \vspace*{2mm}
        \item<3-> Symbolische Analyse kann allein logische Fehler identifizieren
        \vspace*{2mm}
        \item<4-> \textbf{Tamarin} und \textbf{ProVerif} sind populäre Programme zur symbolischen Modellprüfung von Sicherheitsprotokollen
        \vspace*{2mm}
        \item<5-> ProVerif bietet mehrere nützliche Sprachkonzepte für PRINS (Synchronisation, private Tabellen, Ununterscheidbarkeit)
    \end{itemize}
\end{frame}

% PRINS Modellierung
\section{Modellierung von PRINS}

\begin{frame}
    \frametitle{Modellierung von PRINS}
    \framesubtitle{Überblick}

    \begin{itemize}

        \item<1-> Vier Protokollteilnehmer: SEPP A, SEPP B, IPX A, IPX B
        \vspace*{2mm}
        \item<2-> Angreifer implizit simuliert sowie ausgewählte Protokoll- verletzungen durch maliziöse IPX Provider nachgestellt
        \vspace*{2mm}
        \item<3-> Modellierte Prozesse:
        \begin{itemize}
            \item<3-> N32-c Sitzungsaufbau und Fehlersignalisierung
            \item<3-> N32-f Nachrichtenversand und -empfang
            \item<3-> IPX Nachrichtenänderungen
        \end{itemize}
        \vspace*{2mm}
        \item<4-> 12 Sicherheitsanforderungen in ProVerif Abfragen formalisiert
    \end{itemize}
\end{frame}

\begin{frame}
    \frametitle{Modellierung von PRINS}
    \framesubtitle{Abstraktionen und Defizite}

    \begin{itemize}
        \item<1-> Nachrichtenrouting und IPX-Provider Interna
        \vspace*{2mm}
        \item<2-> Protection Policies als statische, eindimensionale Bitstrings
        \vspace*{2mm}
        \item<3-> Transport Layer Security als perfekt sicherer Kanal
        \begin{itemize}
            \item[$\rightarrow$]<3-> Erfolgreiche Authentifizierung vorausgesetzt
            \item[$\rightarrow$]<3-> Export von Schlüsselmaterial mittels privater Funktion
        \end{itemize}
        \vspace*{2mm}
        \item<4-> Fehlende Invalidierung von Sitzungskontexten
        \begin{itemize}
            \item[$\rightarrow$]<4-> Unbegrenzte Validität sobald N32-c Sitzungsaufbau erfolgt
        \end{itemize}
    \end{itemize}
\end{frame}

% Verifikation mit ProVerif
\section{Verifikation mit ProVerif}

\begin{frame}
    \frametitle{Verifikation mit ProVerif}
    \framesubtitle{Ergebnisse der ProVerif Abfragen}

    \begin{itemize}
        \item[+]<1-> Sitzungsaufbau (Context ID, Keys) erfolgreich verifiziert
        \vspace*{2mm}
        \item[+]<2-> Vertraulichkeitsanforderungen erfolgreich verifiziert
        \vspace*{2mm}
        \item[$\sim$]<3-> Illegitime IPX Änderungen nur teilweise erkannt:
        \begin{itemize}
            \item[+]<3-> Ablehnen illegitimer Nachrichtenänderungen
            \item[+]<3-> Ablehnen von Änderungen unbekannter     IPX Provider
            \item[--]<4-> Entfernen hinzugefügter Nachrichtenänderungen
            \item[--]<4-> Wiedereinspielen aufgezeichneter Nachrichtenänderungen
        \end{itemize}
        \vspace*{2mm}
        \item[--]<5-> Abfragen bzgl. Integrität und Authentizität von N32-f~Nachrichten sowie Nachrichtenänderungen widerlegt
        \vspace*{2mm}
        \item[--]<6-> Überprüfung von Ununterscheidbarkeit nicht möglich
    \end{itemize}
\end{frame}

\begin{frame}
    \frametitle{Verifikation mit ProVerif}
    \framesubtitle{Herausforderungen während der Verifikation}

    \begin{itemize}
        \item<1-> Eingeschränkte Möglichkeiten zur Identifikation von Fehlern
        \vspace*{2mm}
        \item<2-> Potenziell ausbleibende Terminierung von ProVerif
        \vspace*{2mm}
        \item<3-> Zeitintensives Rendering von Angriffssequenzen
        \vspace*{2mm}
        \item<4-> Fehlende Optimierung für Multiprozessorsysteme
    \end{itemize}
\end{frame}

% Fazit
\section{Fazit und Ausblick}

\begin{frame}
    \frametitle{Fazit}

    \begin{itemize}
        \item[\checkmark]<1-> Zentrale Aspekte von PRINS in ProVerif nachgebildet
        \vspace*{2mm}
        \item[\checkmark]<2-> Diverse Unklarheiten in 3GPP Spezifikation identifiziert
        (Definiertes Bedrohungsmodell, Sicherheitsanforderungen)
        \vspace*{2mm}
        \item[\checkmark]<3-> Erfolgreiche Verifikation aller Vertraulichkeitsanforderungen
        \vspace*{2mm}
        \item[?]<4-> Widerlegte ProVerif Abfragen nicht endgültig verifizierbar
        \vspace*{2mm}
        \item[?]<5-> Keine Möglichkeit Korrektheit des Modells zu überprüfen
        \vspace*{2mm}
        \item[$\rightarrow$]<6-> Keine konkreten Schwachstellen in PRINS nachweisbar
    \end{itemize}
\end{frame}

\begin{frame}
    \frametitle{Ausblick}
    Mögliche Folgeaktivitäten:
    \vspace*{2mm}
    \begin{itemize}
        \item Verfeinern der Verifikation (Abfragen und Lösungsstrategien)
        \vspace*{2mm}
        \item Tiefergehende Analyse der textbasierten Ausgabe
        \vspace*{2mm}
        \item Optimierung des Rendering-Algorithmus gerichteter Graphen
    \end{itemize}
\end{frame}

\begin{frame}
    \centering
    \huge
    Vielen Dank für\\Ihre Aufmerksamkeit
\end{frame}

\end{document}