\section{Methodik}

Voraussetzung für eine erfolgreiche formale Verifikation sind in jedem Fall solide Kenntnisse über die Grundlagen temporaler Logik und der Modellprüfung.
Daher steht zu Beginn ein umfassendes Studium der einschlägigen Literatur.
Anschließend soll (soweit möglich) ein systematischer Vergleich existierender Verifikationsprogramme für Sicherheitsprotokolle aufgestellt werden, um eine fundierte Entscheidung für die zu verwendende Modellierungssprache treffen zu können.
Basis dafür bilden praktische Gesichtspunkte zur Wahl eines geeigneten Formalismus, welche unter anderem von Seshia et al. \cite{seshia2018model} beschrieben werden.
Nähere Informationen über die bestehenden Verifikationsprogramme werden den jeweiligen Dokumentationen sowie verwandten Arbeiten auf diesem Gebiet, einige davon exemplarisch benannt in Abschnitt~\ref{related}, entnommen.

Daraufhin schließt sich die Formalisierung des 5G Inter-Operator-Signalisierungsprotokolls an.
Von besonderer Bedeutung ist dabei die korrekte Repräsentation der laut Spezifikation geforderten Sicherheitsziele und des Angreifermodells.
Letzteres ist in den Dokumenten der \gls{3gpp} nicht explizit beschrieben, daher sollen Informationen darüber im Rahmen von Experteninterviews zusammengetragen werden.
Eventuell getroffene Annahmen über den korrekten Ablauf des Protokolls, welche ggf. nötig sind um die Vollständigkeit des Modells zu gewährleisten, werden dabei explizit als solche gekennzeichnet.
Das erstellte Modell wird mittels des zuvor gewählten Programms gegen die Spezifikation auf Korrektheit geprüft.
Sollten sich durch die computergestützte Verifikation bis dato unbehandelte Fehlerszenarien identifizieren lassen, so wird eine Ursachenanalyse durchgeführt.
Diese ist unerlässlich, um abschließend möglicher Gegenmaßnahmen auf Spezifikations- oder Implementationsebene aufzuzeigen.
Dieser letzte Schritt ist in gewissem Maße abhängig von Erfahrungswerten des Autors aus der Mobilfunkstandardisierung, da die in der \gls{3gpp} definierten Protokolle nicht immer bis ins letzte Detail beschrieben werden und mitunter ein gewisses Sicherheitsniveau der individuellen Implementierung vorausgesetzt wird. 

Nachfolgend ist ein erster Gliederungsentwurf der Master-Thesis, basierend auf dem oben beschriebenen Vorgehen, dargestellt:

\begin{enumerate}[label=\arabic*]
    \item Einleitung
    \begin{enumerate}[label=1.\arabic*]
        \item Problemstellung und Erkenntnisinteresse
        \item Zielsetzung und Abgrenzung
        \item Methodik
    \end{enumerate}
    \item Theoretische Grundlagen
    \begin{enumerate}[label=2.\arabic*]
        \item Funktionsweise des 5G Inter-Operator-Protokolls
        \begin{enumerate}[label=2.1.\arabic*]
            \item Sicherheitsziele
            \item Angreifermodell
        \end{enumerate}
        \item Formale Verifikation von Sicherheitsprotokollen
        \begin{enumerate}[label=2.2.\arabic*]
            \item Grundlagen der formalen Modellprüfung
            \item Vergleich existierender Formalismen \& Werkzeuge 
        \end{enumerate}
    \end{enumerate}
    \item Formale Verifikation
    \begin{enumerate}[label=3.\arabic*]
        \item Transposition der 3GPP-Spezifikation
        \item Verifikation erwarteter Sicherheitsziele
        \item Fehler- und Ursachenanalyse
        \item Identifikation möglicher Verbesserungen
    \end{enumerate}
    \item Abschließende Betrachtungen
    \begin{enumerate}[label=4.\arabic*]
        \item Fazit
        \item Ausblick
    \end{enumerate}
\end{enumerate}