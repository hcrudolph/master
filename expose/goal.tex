\section{Zielsetzung und Abgrenzung}

Im Rahmen dieser Master-Thesis soll das 5G Inter-Operator-Signalsierungsprotokoll, dessen zentraler Bestandteil \gls{als} darstellt, mittels formaler Verifikation überprüft werden.
Dazu zählt in einem ersten Schritt die Untersuchung inwiefern sich bestehende Modelle für die Formalisierung der spezifizierten Funktionalität eignen.
Daraufhin folgt, in Vorbereitung der Anwendung des gewählten Verifikationsprogramms, das Überführen der textuellen Beschreibung in eine entsprechende formale Repräsentation.
Bereits während dieser Transposition können Spezifikationslücken bzw. Unklarheiten in den \gls{3gpp}-Dokumenten identifiziert werden.
Ziel der sich anschließenden programmgestützten Analyse ist sowohl das Verifizieren von Korrektheitsbeweisen geforderter Sicherheitsziele als auch das Erkennen potenzieller Fehlerszenarien und Ableiten entsprechender Korrekturen, um diese zu verhindern.
Die konkreten Forschungsfragen lassen sich daher wie folgt formulieren:

\begin{enumerate}[label=(\arabic*)]
    \item Lässt sich das 5G Inter-Operator-Signalisierungsprotokoll mit bestehenden Methoden zur formalen Verifikation vollumfänglich beschreiben?
    
    \item Inwiefern lassen sich auf Basis der aufgestellten formalen Repräsentation Unklarheiten bzw. Fehler in der Protokollspezifikation identifizieren?
    
    \item Welche möglichen Korrekturen auf Spezifikations- oder Implementationsebene wären zu definieren, um potenzielle Fehler aus (2) zu verhindern?
\end{enumerate}

Die Verifikation beschränkt sich dabei ausschließlich auf die in \gls{ts} 33.501~\cite{TS33501} und \gls{ts} 29.573~\cite{TS29573} neu definierte Kommunikation.
Teilprotokolle, die extern beschrieben und mitunter selbst schon mittels formaler Methoden überprüft wurden, wie z.B. \gls{tls}, werden dabei nicht genauer betrachtet und abstrahiert als sichere Kanäle behandelt.
Des Weiteren ist explizit die Anwendung einer bereits beschriebenen Modellierungssprache bzw. eines existierenden Verifikationsprogramms vorgesehen.
Falls die zur Verfügung stehenden Ressourcen, näher beschrieben in Abschnitt \ref{related}, nicht ausreichen, um das 5G Inter-Operator-Signalisierungsprotokoll hinreichend zu beschreiben, soll detailliert werden woran die Formalisierung scheitert und wie bestehende Programme ggf. zu erweitern wären.
Es soll hingegen kein komplett neues Modellierungsverfahren zur formalen Überprüfung erarbeitet werden.